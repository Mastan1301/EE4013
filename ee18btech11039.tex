\documentclass[journal,12pt,twocolumn]{IEEEtran}

\usepackage{setspace}
\usepackage{gensymb}
\singlespacing
\usepackage[cmex10]{amsmath}

\usepackage{amsthm}

\usepackage{mathrsfs}
\usepackage{txfonts}
\usepackage{stfloats}
\usepackage{bm}
\usepackage{cite}
\usepackage{cases}
\usepackage{subfig}

\usepackage{longtable}
\usepackage{multirow}

\usepackage{enumitem}
\usepackage{mathtools}
\usepackage{steinmetz}
\usepackage{tikz}
\usepackage{circuitikz}
\usepackage{verbatim}
\usepackage{tfrupee}
\usepackage[breaklinks=true]{hyperref}
\usepackage{graphicx}
\usepackage{tkz-euclide}
\usetikzlibrary{shapes,arrows}

\usetikzlibrary{calc,math}
\usepackage{listings}
    \usepackage{color}                                            %%
    \usepackage{array}                                            %%
    \usepackage{longtable}                                        %%
    \usepackage{calc}                                             %%
    \usepackage{multirow}                                         %%
    \usepackage{hhline}                                           %%
    \usepackage{ifthen}                                           %%
    \usepackage{lscape}     
\usepackage{multicol}
\usepackage{chngcntr}

\DeclareMathOperator*{\Res}{Res}

\renewcommand\thesection{\arabic{section}}
\renewcommand\thesubsection{\thesection.\arabic{subsection}}
\renewcommand\thesubsubsection{\thesubsection.\arabic{subsubsection}}

\renewcommand\thesectiondis{\arabic{section}}
\renewcommand\thesubsectiondis{\thesectiondis.\arabic{subsection}}
\renewcommand\thesubsubsectiondis{\thesubsectiondis.\arabic{subsubsection}}


\hyphenation{op-tical net-works semi-conduc-tor}
\def\inputGnumericTable{}                                 %%

\lstset{
%language=C,
frame=single, 
breaklines=true,
columns=fullflexible
}
\begin{document}


\newtheorem{theorem}{Theorem}[section]
\newtheorem{problem}{Problem}
\newtheorem{proposition}{Proposition}[section]
\newtheorem{lemma}{Lemma}[section]
\newtheorem{corollary}[theorem]{Corollary}
\newtheorem{example}{Example}[section]
\newtheorem{definition}[problem]{Definition}

\newcommand{\BEQA}{\begin{eqnarray}}
\newcommand{\EEQA}{\end{eqnarray}}
\newcommand{\define}{\stackrel{\triangle}{=}}
\bibliographystyle{IEEEtran}
\raggedbottom
\setlength{\parindent}{0pt}
\providecommand{\mbf}{\mathbf}
\providecommand{\pr}[1]{\ensuremath{\Pr\left(#1\right)}}
\providecommand{\qfunc}[1]{\ensuremath{Q\left(#1\right)}}
\providecommand{\sbrak}[1]{\ensuremath{{}\left[#1\right]}}
\providecommand{\lsbrak}[1]{\ensuremath{{}\left[#1\right.}}
\providecommand{\rsbrak}[1]{\ensuremath{{}\left.#1\right]}}
\providecommand{\brak}[1]{\ensuremath{\left(#1\right)}}
\providecommand{\lbrak}[1]{\ensuremath{\left(#1\right.}}
\providecommand{\rbrak}[1]{\ensuremath{\left.#1\right)}}
\providecommand{\cbrak}[1]{\ensuremath{\left\{#1\right\}}}
\providecommand{\lcbrak}[1]{\ensuremath{\left\{#1\right.}}
\providecommand{\rcbrak}[1]{\ensuremath{\left.#1\right\}}}
\theoremstyle{remark}
\newtheorem{rem}{Remark}
\newcommand{\sgn}{\mathop{\mathrm{sgn}}}
\providecommand{\abs}[1]{\left\vert#1\right\vert}
\providecommand{\res}[1]{\Res\displaylimits_{#1}} 
\providecommand{\norm}[1]{\left\lVert#1\right\rVert}
%\providecommand{\norm}[1]{\lVert#1\rVert}
\providecommand{\mtx}[1]{\mathbf{#1}}
\providecommand{\mean}[1]{E\left[ #1 \right]}
\providecommand{\fourier}{\overset{\mathcal{F}}{ \rightleftharpoons}}
%\providecommand{\hilbert}{\overset{\mathcal{H}}{ \rightleftharpoons}}
\providecommand{\system}{\overset{\mathcal{H}}{ \longleftrightarrow}}
	%\newcommand{\solution}[2]{\textbf{Solution:}{#1}}
\newcommand{\solution}{\noindent \textbf{Solution: }}
\newcommand{\cosec}{\,\text{cosec}\,}
\providecommand{\dec}[2]{\ensuremath{\overset{#1}{\underset{#2}{\gtrless}}}}
\newcommand{\myvec}[1]{\ensuremath{\begin{pmatrix}#1\end{pmatrix}}}
\newcommand{\mydet}[1]{\ensuremath{\begin{vmatrix}#1\end{vmatrix}}}
\numberwithin{equation}{subsection}
\makeatletter
\@addtoreset{figure}{problem}
\makeatother
\let\StandardTheFigure\thefigure
\let\vec\mathbf
\renewcommand{\thefigure}{\theproblem}
\def\putbox#1#2#3{\makebox[0in][l]{\makebox[#1][l]{}\raisebox{\baselineskip}[0in][0in]{\raisebox{#2}[0in][0in]{#3}}}}
     \def\rightbox#1{\makebox[0in][r]{#1}}
     \def\centbox#1{\makebox[0in]{#1}}
     \def\topbox#1{\raisebox{-\baselineskip}[0in][0in]{#1}}
     \def\midbox#1{\raisebox{-0.5\baselineskip}[0in][0in]{#1}}
\vspace{3cm}
\title{EE4013 Assignment-1}
\author{Shaik Mastan Vali - EE18BTECH11039}
\maketitle
\newpage
\bigskip
\renewcommand{\thefigure}{\theenumi}
\renewcommand{\thetable}{\theenumi}
Download all codes from 
\begin{lstlisting}
https://github.com/Mastan1301/EE4013/tree/master/assignment-1/codes/
\end{lstlisting}
%
and latex-tikz codes from 
%
\begin{lstlisting}
https://github.com/Mastan1301/EE4013/tree/master/assignment-1/
\end{lstlisting}

\section{Problem}
Consider the following matrix:
\begin{align*}
    R = \begin{bmatrix}
        1 & 2 & 4 & 8 \\
        1 & 3 & 9 & 27 \\
        1 & 4 & 16 & 64 \\
        1 & 5 & 25 & 125
    \end{bmatrix}
\end{align*}
Calculate the absolute value of the Eigen values of $R$.

\section{Solution}
We use the following property -
\begin{align*}
    \prod_{i = 0}^{k - 1} \lambda_{i} = det(R)
\end{align*}
where $\lambda_{i}$ are the Eigen values and $det(.)$ is the determinant operator. \\
To find the determinant, we use the following recurrence relation.
\begin{align*}
    det(R, n) = \sum_{j = 0}^{n - 1} (-1) ^ {j} \times R[0][j] \times det(cof(R, 0, j), n - 1)
\end{align*}
where $cof(i, j)$ is the cofactor matrix of the position $(i, j)$ in the matrix R. \\ \\

The recursion, for our example, is given in the flow chart. \\ \\

\begin{figure}[h!]
	\begin{center}
		\resizebox{\columnwidth/1}{!}{\tikzstyle{rblock} = [rectangle, draw, text width=15em, text centered, inner sep=0pt, minimum height=2em, rounded corners]
\tikzstyle{line} = [draw, -latex']
\tikzstyle{arrow} = [thick,->,>=stealth]

\begin{tikzpicture}[node distance = 5em, auto]
    \node (node1) [rblock] {$det(X, 4)$};
    \node (node2) [rblock, below of = node1] {$X[0][0] * det(cof(X, 0, 0), 3) - X[0][1] * det(cof(X, 0, 1), 3) + X[0][2] * det(cof(X, 0, 2), 3) - X[0][3] * det(cof(X, 0, 3), 3)$};
    \draw [arrow] (node1) -- (node2);
\end{tikzpicture}}
	\end{center}
	\caption{The flow of recursion}
	\label{fig:fig1}
\end{figure}

\\ \\ \\ To understand the process, consider $C_1 = cof(R, 0, 0)$. \\ \\

\begin{figure}[h!]
	\begin{center}
		\resizebox{\columnwidth/1}{!}{\tikzstyle{rblock} = [rectangle, draw, text width=15em, text centered, inner sep=0pt, minimum height=2em, rounded corners]
\tikzstyle{line} = [draw, -latex']
\tikzstyle{arrow} = [thick,->,>=stealth]

\begin{tikzpicture}[node distance = 5em, auto]
    \node (node3) [rblock] {$det(C_1, 3)$};
    \node (node4) [rblock, below of = node3] {$C_1[0][0] * det(cof(C_1, 0, 0), 2) - C_1[0][1] * det(cof(C_1, 0, 1), 2) + C_1[0][2] * det(cof(C_1, 0, 2), 2)$};
    \draw [arrow] (node3) -- (node4);
\end{tikzpicture}}
	\end{center}
	\caption{A $3 \times 3$ sub-problem}
	\label{fig:fig2}
\end{figure}

Now, consider $C_2 = cof(C_1, 0, 0)$. 

\begin{figure}[h!]
	\begin{center}
		\resizebox{\columnwidth/1}{!}{\tikzstyle{rblock} = [rectangle, draw, text width=15em, text centered, inner sep=0pt, minimum height=2em, rounded corners]
\tikzstyle{line} = [draw, -latex']
\tikzstyle{arrow} = [thick,->,>=stealth]

\begin{tikzpicture}[node distance = 5em, auto]
    \node (node5) [rblock] {$det(C_2, 2)$};
    \node (node6) [rblock, below of = node3] {$C_2[0][0] * det(cof(C_2, 0, 0), 1) - C_2[0][1] * det(cof(C_2, 0, 1), 1)$};
    \draw [arrow] (node5) -- (node6);
\end{tikzpicture}}
	\end{center}
	\caption{A $2 \times 2$ sub-problem}
	\label{fig:fig3}
\end{figure}

The determinant of a $1 \times 1$ matrix is the value of the element itself. We use this property as the base case.

\section{Complexity of the algorithm}
\textbf{Time Complexity: } At each stage of recursion, there are $O(n)$ recursive calls. In each of these function calls, we compute the co-factor matrix in $O(n ^ 2)$ time. So, the total time complexity is $O(n ^ 3)$.
\end{document}

