\begin{frame}{Solution-2}
    \section{Solution-2}
    In this solution, we use the QR-algorithm to compute the Eigen values. The algorithm is as follows-
    \begin{itemize}
        \item Apply the QR decomposition so that
                \begin{align*}
                    X = Q_0R_0
                \end{align*}
                where $Q_0$ is an orthogonal matrix and $R_0$ is an upper-triangular matrix.
        \item Compute 
              \begin{align*}
                    E_0 = Q_0^{-1} X Q_0 = Q_0 ^ T X Q_0
                \end{align*} 
                Note that the Eigen values of $X$ and $E_0$ are the same.
        \item Next, we decompose the $E_0$ and repeat this procedure till we obtain a triangular matrix. The Eigen values of a triangular matrix are given by the diagonal entries of the matrix.
    \end{itemize}
\end{frame}

\begin{frame}{Householder Transformation}
    The reflection hyperplane can be defined by a unit vector $u$ that is orthogonal to the hyperplane, which is its normal vector. The reflection of a vector $x$ about this hyperplane is the linear transformation:
    \begin{align*}
        y &= x - 2 (u^T x) u \\
        &= x - 2 u (u ^T x) \\
        &= x - 2 (u u ^T) x \\
        &= (I - 2 u u^T) x
    \end{align*}
    The matrix constructed from this transformation can be expressed in terms of an outer product as:
    \begin{align*}
        H = H=I-2 u u^{T}
    \end{align*}
    is called as the \textbf{Householder matrix}.
\end{frame}

\begin{frame}{QR decomposition}
    We compute the QR decomposition using the Householder transformation.
    Multiplying a given vector $x$, for example the first column of matrix $X$, with the Householder matrix $H$, which is given as
    \begin{align*}
        H=I-\frac{2}{u^{T} u} u u^{T}
    \end{align*}
    
    reflects $x$ about a plane given by its normal vector $u$. When the normal vector of the plane $u$ is given as
    \begin{align*}
        u=x-\|x\|_{2} e_{1}
    \end{align*}
    then the transformation reflects $x$ onto the first standard basis vector
    \begin{align*}
            e_{1}=\left[\begin{array}{llll}
            1 & 0 & 0 & \ldots
        \end{array}\right]^{T}
    \end{align*}
    which means that all entries but the first become zero.
\end{frame}

\begin{frame}{Algorithm}
    In general, if we take $u=x-s\|x\| e_{1}$ where $s=\pm 1$ and $v=u /\|u\|$ then
    $$
    H x=\left(I-2 \frac{u u^{T}}{u^{T} u}\right) x=s\|x\| e_{1} .
    $$
    Let us first verify that this works:
    \begin{align*}
    u^{T} x &=\left(x-s\|x\| e_{1}\right)^{T} x \\
    &=\|x\|^{2}-s x_{1}\|x\| \\
    u^{T} u &=\left(x-s\|x\| e_{1}\right)^{T}\left(x-s\|x\| e_{1}\right) \\
    &=\|x\|^{2}-2 s x_{1}\|x\|+\|x\|^{2}\left\|e_{1}\right\|^{2} \\
    &=2\left(\|x\|^{2}-s x_{1}\|x\|\right)=2 u^{T} x \\
    H x &=x-2 u \frac{u^{T} x}{u^{T} u}=x-u=s\|x\| e_{1} .
    \end{align*}
\end{frame}

\begin{frame}{Algorithm}
    As a byproduct of this calculation, note that we have
    $$
    u^{T} u=-2 s\|x\| u_{1} = -2 s\|x\| (x_1 - s\|x\|)
    $$
    Hence, to avoid numerical cancellation errors, we take $s = -sign(x_1)$.\\
    $$\implies Hx = -sign(x_1) \|x\| e_1$$
    First, we multiply $X$ with the Householder matrix $H_{1}$ we obtain when we choose the first matrix column for $X$. This results in a matrix $H_{1} X$ with zeros in the left column (except for the first row).
\end{frame}

\begin{frame}{Algorithm}
    $$
    H_{1} X=\left[\begin{array}{cccc}
    -sign(X_{11})\alpha_{1} & \star & \cdots & \star \\
    0 & & & \\
    \vdots & & X^{\prime} & \\
    0 & & &
    \end{array}\right]
    $$
    where $\alpha_1$ is the 2-norm of the first column of $X$.\\
    This can be repeated for $X^{\prime}$ (obtained from $H_{1} X$ by deleting the first row and first column), resulting in a Householder matrix $H_{2}^{\prime} .$ Note that $H_{2}^{\prime}$ is smaller than $H_{1}$. Since we want it really to operate on $H_{1} X$ instead of $X^{\prime}$ we need to expand it to the upper left, filling in a 1 , or in general:
    $$
    H_{k}=\left[\begin{array}{cc}
    I_{k-1} & 0 \\
    0 & H_{k}^{\prime}
    \end{array}\right]
    $$
\end{frame}

\begin{frame}{Algorithm}
This is how we can, column by column, remove all sub-diagonal elements of $A$ and thus transform it into $R$.
$$H_{n} \ldots H_{3} H_{2} H_{1} X=R$$
The product of all the Householder matrices $H$, for every column, in reverse order, will then yield the orthogonal matrix $Q$.
$$
H_{1} H_{2} H_{3} \ldots H_{n}=Q
$$
\end{frame}

\begin{frame}{Complexity of the algorithm}
    The QR decomposition step takes $O(n ^ 3)$ computations, because the number of multiplications at each iteration is $O(n^2)$, and there are $O(n)$ iterations. \\
    The matrix convergence is achieved in $O(\max_{i, j} |\lambda_i / \lambda_j| ^ n)$ steps. Hence the total time complexity is $O((\max_{i, j} |\lambda_i / \lambda_j| ^ n) \times n ^ 3)$.
\end{frame}